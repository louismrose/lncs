\section{Introduction}

A central point in the value proposition of BPM suites is that a business process model can be used by different stakeholders for different purposes in the BPM life cycle. It can be used by a business analyst to document, analyze or communicate a process. Technical architects and developers can use a process model to implement the business process on a particular process engine. These are perhaps the two most prominent uses of a process model, but a process model can also be used by a business analyst to visualize monitoring data from the live system, or by an end user of the system, i.e., a process participant, to understand the context of his or her participation in the process.

These different stakeholders would ideally share a single process model to collaborate and to communicate to each other their interests regarding a particular business process. For example, a business analyst and a technical architect could negotiate process changes through the shared model. The business analyst could initiate process changes motivated by new business requirements, which can then be immediately seen by the technical architect and forms the basis for him to evaluate and implement the necessary changes to the IT system. The technical architect may revise the change because it is not implementable in the proposed form on the existing architecture.
Vice versa, a technical architect can also initiate and communicate process changes motivated from technical requirements, e.g., new security regulations, revised performance requirements, etc. In this way, a truly shared process model can increase the agility of the enterprise.

This appealing vision of a single process model that is shared between stakeholders is difficult to achieve in practice. One practical problem is that, in some enterprises, different stakeholders use different metamodels and/or different tools to maintain their version of the process model. This problem makes it technically difficult to conceptually share `the' process model between the stakeholders (the \emph{\bpmn-\bpel roundtripping problem} is a known example). This technical problem disappears with modern BPM suites and the introduction of \bpmn 2, as this single notation now supports modeling both business and IT-level concerns.

However, there is also an essential conceptual problem. We argue that different stakeholders intrinsically want different \emph{views} onto the same process because of their different concerns and their different levels of abstraction. This is even true for parts that all stakeholders are interested in, e.g., the main behavior of the process.
Therefore we argue that we need separate, stakeholder-specific views of the process that are kept \emph{consistent} with respect to each other. Current tools do not address this problem. Either different stakeholders use different models of the same process, which then quickly become inconsistent, or they use the same process model, which then cannot reflect the needs of all stakeholders.

This problem is a variation of the coupled evolution problem~\cite{HerrmannsdoerferBJ09} and the model synchronization problem~\cite{GieseW09}. Coupled evolution has been studied between metamodels and models but not for process models at different abstraction levels and in the area of model synchronization various techniques have been proposed. Put into this context, our research question is how process views at different abstraction levels can be kept consistent and changes can be propagated in both directions automatically in a way that is aligned with existing studies of requirements from practice. In this paper, we address this problem, present detailed requirements and a design to synchronize process views on different abstraction levels. The challenge for a solution arises from an interplay of a variety of possible interdependent process model changes and their translation between the abstraction levels. We also report on an implementation to substantiate that a solution is indeed technically feasible. An extended version of this paper is available as a technical report~\cite{KVF+TR}.
 
\forget{
The paper is structured as follows. In Sect.~\ref{sec:BITGap}, we explain in detail why we think that the ability for synchronizing different process models views is needed. Although the problem is recognized in academic research, it is not universally accepted in practice after the arrival of BPMN 2.
Based on this motivation, we present detailed requirements for a Shared Process Model in Sect.~\ref{sec:requirements} . We also explain several usage scenarios of the Shared Process Model and discuss key aspects of the consistency between different views. In Sect.~\ref{sec:realization} we introduce our technical realization of the Shared Process Model and report on a prototype. Related work is discussed in Sect.~\ref{sec:related}.} 



