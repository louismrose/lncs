\section{Conclusion and Future Work}\label{sec:conclusion}
In this paper, we have identified and addressed the challenges that emerge from integrating run-time monitoring of complex signatures into the AUTOSAR development process.
The presented continuous model-based development process for security and safety monitors (MBSecMon) is integrated into the AUTOSAR process and incorporates data of the AUTOSAR models.
As shown, it allows the modeling of monitor signatures in a well comprehensible graphical modeling language and the automatic generation of monitors with a low overhead that fulfill the AUTOSAR conventions.
This framework and the generated monitors have been evaluated utilizing an example model provided with the MATLAB suite, for which AUTOSAR code was generated and integrated into an AUTOSAR environment.
With the presented approach, we have overcome the lack of support for complex monitoring in the AUTOSAR tool chains.
It is applicable to white-box (source code) and black-box (binary) components, as communication between components is intercepted directly at their port interface by instrumentation techniques shown in~\cite{Piper2012}.

In the future, an evaluation in a larger AUTOSAR project with more complex interactions is planned.
% Find ich eher verwirrend als Leser, der von den verschiedenen Kommunikationsparadigmen in AUTOSAR keine Ahnung hat.
%Additionally, the prototypical implementation can be extended to support the client-server communication of the Virtual Functional Bus between the SW-Cs. 
It has to be evaluated if the MBSecMon process can be used for Logical Program Flow Monitoring~\cite{AUTOSAR_WDM_SPEC}, eventually with another source specification language such as UML2 activity diagrams or MPNs directly. 