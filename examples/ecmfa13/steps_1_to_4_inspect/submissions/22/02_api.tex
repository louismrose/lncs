Our work targets at voice and key-strokes driven telephony services. These services normally run on the
infrastructure of the service provider. Subsection~\ref{sec:JSLEE} reviews a standard (JAIN SLEE) that can be used for building
this kind of applications, while subsection~\ref{sec:umbra-framework} introduces a Java framework built atop JAIN SLEE
which provides higher-level abstractions and true portability across SLEE implementations.

%---------------------------------------------------------
\subsection{JAIN SLEE}\label{sec:JSLEE}
%---------------------------------------------------------

JAIN SLEE is a standard by the Java Community Process that describes
a Service Logic and Execution Environment (SLEE) architecture~\cite{JAINSLEE}.
This architecture defines a component model for structuring the logic of communications applications
as a collection of object-oriented components (called Service Building Blocks, SBBs), which can be composed into
services. The SLEE architecture also defines the contract between these components
and the container that will host them at runtime. 
JAIN SLEE applications are event-driven, which means that methods of the application are invoked when
suitable events arrive. In this way, each SBB to be deployed in the SLEE identifies the event types 
that accepts, and defines event handler methods with code to process such event types. 

The framework provides an API for handling events, resources and connections, facilities like timers 
and alarms, and standard interfaces to be implemented by SBBs. Still, the API is low-level, and 
service developers would benefit from higher-level abstractions, tailored to voice-driven telephony 
services, as explained in the next subsection.

%-------------------------------------------------------------------------------
\subsection{The Umbra Framework}\label{sec:umbra-framework}
%-------------------------------------------------------------------------------

The {\em Umbra framework} hides the low-level details of JAIN SLEE to enhance performance and simplify 
the development of telephony services. It masks the JAIN SLEE components behind a simpler Java API that offers 
enhanced, scalable, carrier-grade performance. Another benefit is portability. As different providers offer 
different implementations of the standard, there may be JAIN SLEE compliant applications that do not run 
on every SLEE container. Moreover, when migrating an application from one vendor to another, parts
of its code may need to be re-written to ensure smooth porting and compatibility. With the {\em Umbra 
framework}, the code works across JAIN SLEE platforms from the main vendors without any recoding.

The framework enables building applications mixing both web and telephony services.
While JAIN SLEE can deal with low-level protocol issues at the back-end, a J2EE environment can provide a 
front-end for Internet services. Hence, {\em Umbra} enables the SLEE to offer web services the J2EE world can interact with.
%so that the J2EE world can interact with the network capabilities exposed by the SLEE.

Fig.~\ref{fig:UmbraStructure} shows the typical structure of a service built with the {\em Umbra framework}. 
The upper part shows only the most relevant interfaces provided by the framework. The lower part (package 
{\em application}) presents a schema of the classes and interfaces that programmers have to develop. 
As we will see, {\em Umbra} is based on the definition of suitable event types and listeners for such events. The
listeners contain call-back methods, invoked upon the reception of the events, which need to be programmed
by the service developers.

\figeps{UmbraStructure}{0.42}{Structure of an application using the {\em Umbra framework} (simplified).}

A service is started upon the reception of the {\em onBootstrap} event. Hence, the developer needs to 
implement code to react to this event in the {\em BootStrapEventListener}. Normally, 
this code includes loading the needed resources and register event listeners  
through a {\em ServiceLocator}, especially {\em SessionRequest} events, which are events triggered by 
the SLEE container when the network requests a new session. Then, the service waits for incoming events, which
may trigger specific actions like playing a message, recording a message, soliciting the user to press a key,  
and so on. These actions are supported by an {\em IVRServer} (a media server), which is a resource
that needs to be identified upon bootstrap. The service will receive an event notification upon completion of the 
actions, as declared in the {\em IvrServerEventListener}. For example, the {\em onPlayed} method will be 
called upon completion of a {\em play} action, which plays a voice message.

Practice has shown that a suitable organization for event-driven applications is through state machines.
Hence, a typical programming idiom for services built with the {\em Umbra} framework is the {\em State 
pattern}~\cite{Gamma} in order to describe the different execution states of the service, the possible 
incoming events, the actions to be performed upon the arrival of events, and the state changes. This way,
services normally define an interface {\em State} declaring all possible events, while the abstract class 
{\em AbstractState} defines default empty implementations for the event handlers. Therefore, developers 
have to create a subclass of {\em AbstractState} per application state ({\em State1} in Fig.~\ref{fig:UmbraStructure}), 
and override the methods for the events accepted by the state.

This organization is a Java implementation of a natural way of designing services as state machines. 
However, this style of programming is not enforced by the {\em Umbra framework}
even though it is common practice. Hence, we decided to provide developers with a higher level representation of this pattern, closer
to the abstractions of state machines. This way, the gap from design to implementation would be smaller. 
Moreover, this representation would facilitate the communication with customers, most frequently
non-technical people. The next section introduces a Domain-Specific Visual Language that helps in describing 
services at a higher level of abstraction.


