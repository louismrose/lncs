\section{Related Work}
\label{sec:relatedWork}

The AUTOSAR standard specifies several run-time monitoring mechanisms that are provided by the Operating System (OS) and the Watchdog Manager (WdM).
In the \emph{Specification of Operating System} \cite{AUTOSAR_OS_SPEC}, three monitoring mechanisms for the detection of timing faults at run-time, i.e., tasks or interrupts missing their deadline, are considered.
\emph{Execution time protection} monitors the execution budget of tasks and category 2 Interrupt Service Routines (ISRs), in order to guarantee that the execution time does not exceed a statically configured upper bound.
\emph{Locking time protection} supervises the time that a task or category 2 ISR can hold a resource, including the suspension time of OS interrupts or all interrupts of the system.
This is done to prevent priority inversions and to recover from potential deadlocks.
The third monitoring mechanism is \emph{inter-arrival time protection}, which controls the time between successive activations of tasks and the arrival of category 2 ISRs.
These mechanisms monitor the system at the task level and are not suited to implement control flow or data flow monitoring.
The configuration is done at the OS level and does not factor the system view of the model.

The \emph{Specification of Watchdog Manager} \cite{AUTOSAR_WDM_SPEC} provides three monitoring mechanisms that are complementary to those offered by the AUTOSAR OS.
All of the implemented mechanisms are based on checkpoints that report to the watchdog manager (WdM) when they are reached.
Supervised functions have to be instrumented with calls to the watchdog, which verifies at run-time the correct transition between two checkpoints as well as the timing of the checkpoint transitions.
For \emph{alive supervision}, the WdM periodically checks if the checkpoints of a supervised entity have been reached within the given limits, in order to detect if a supervised entity is run too frequently or too rarely.
For the supervision of aperiodic or episodical events that have individual constraints on the timing between checkpoints, the WdM offers \emph{deadline supervision}.
In this approach, the WdM checks if the execution time of a given block of a supervised entity is within the configured minimum and maximum bound.
The third mechanism that the WdM provides is \emph{logical monitoring}, which focuses on the detection of control flow errors, which occur if one or more program instructions are processed either in an incorrect sequence or are not processed at all.
This approach resembles the work of Oh et al. \cite{Oh2002}, in which a graph representation of a function is constructed by dividing the function into \emph{basic blocks} at each (conditional) branch.
Basic blocks are represented by nodes, whereas legal branches are represented by arcs that connect the nodes. 
Whenever a new basic block is entered, the monitor verifies that the taken branch was legal, according to the graph representation.

Except for control flow monitoring, the monitoring mechanisms that AUTOSAR currently specifies are only suitable for monitoring timing properties at a low level of abstraction. 
None of the approaches is configured on the level of the AUTOSAR system model, which offers the developer an intuitive and integrated view on the system.
Apart from the monitoring services offered by AUTOSAR, few research has covered this area so far.
One exception are two articles by Cotard et al. \cite{Cotard2012a,Cotard2012b}, in which the authors describe a monitoring approach for the synchronization of tasks on multi-core hardware platforms. 
In their approach, dependencies between tasks are first modeled as a finite state machine (FSM).
The FSM is then translated into a linear temporal logic (LTL) specification, from which Moore machines and, subsequently, C code is generated.
Their primary focus is on synchronization and not on control or data flow monitoring.
%Today, in industry mostly hand programmed watchdog monitors are used that are programmed directly in the target language (C).
%This allows individual monitors for every purpose, but is not well integrated into the AUTOSAR process.
%Conventions, such as naming conventions of AUTOSAR, have to be kept by the programmer and errors in the monitoring code are possible.
%
%To reduce the risk of errors in monitors a model-based specification eases the modeling for the programmer, and the specification gets more readable.
%There are different modeling languages for signatures that can be used for communication and program flow monitoring. 
%EDL~\cite{Schmerl2006}, 
%simple PN languages for monitors in embedded systems~\cite{Frankowiak2009} 
%These languages are only usable for small monitor definitions.
%A framework that is similar to our MBSecMon process is the Monitoring Oriented Programming framework (MOP) \cite{Meredith2011}. Is based on Java.
%
%Are they integrated in the AUTOSAR development Process?

%Support of (Windowed) Watchdog Concept\footnote{http://www.autosar.org/download/R2.0/AUTOSAR\_SWS\_WatchdogManager.pdf}?
%\begin{itemize}
%\item Watchdog manager located n Basic Software, uses hardware watchdog timers over the drivers and accesses the software components via RTE.
%\item allows for alive supervision of single or multiple sw-components (sw-component proves aliveness by updating counter) and non-alive supervision (manager triggers periodically the sw-components)
%\end{itemize}  
%
%Communication Watchdog for periodic messages~\cite{AUTOSAR2011}\\\\
%
%\begin{description}
%  \item[!] Program Flow Monitoring
%  \item[?] Modeling with LSCs (maybe in Section~\ref{sec:mbsecmonFramework})
%  \item[?] Code Generation (maybe in Section~\ref{sec:mbsecmonFramework})
%  \item[?] Instrumentation
%\end{description}

% Another approach is ... TODO: enter healing paper here - The paper "A framework for health monitoring of automotive electrical and electronic control systems" does explicitely not address AUTOSAR. Therefore not really well suited as related work.


% In our process, the expressive and compact Live Sequence Chart descriptions are transformed to a more explicit representation of their semantics represented as Monitor Petri nets.
% These are then used for code generation.
% Mostly, transformations from different source models to Petri nets and temporal logics are performed to analyze and verify the correctness of the source model.
% Amorim et al. \cite{Amorim2006} has defined a transformation from LSC to Colored Petri nets (CPN) to analyze the LSC definitions by simulating the CPNs.
% Fernandez et al. \cite{Fernandes2007} transforms use cases that are specified by sequence diagrams (SD) to CPN with the goal to obtain an executable model that is easier to understand than SDs.
% Therefore, his approach makes extensive use of the ML language, a functional programming language, and hierarchical structures to get ``readable'' results.
In our MBSecMon process, an extended version of the expressive and compact Live Sequence Charts is used as signature specification language. 
Kumar et al. \cite{Kumar2009} specifies protocols with LSCs and transforms them for verification to temporal logic formulas. 
Thereby, complex LSC specifications lead to an explosion of the formula and are, therefore, with LSCs as specification language not suitable as intermediate language for the code generation for embedded systems.

Besides these transformations to LTL, there are some approaches that use Petri nets directly for the specification of monitors.
Frankowiak et al. \cite{Frankowiak2009} uses Petri nets to specify low cost process monitors on a micro controller.
He enhances regular Petri nets by token generators and special end places (bins).
Additionally, subnets are linked by a control net.
For complex monitor specifications, these nets get relatively large compared to MPNs, whose semantics include an implicit evaluation logic when an event does not match to the specified sequence. 
The MPNs are tailored to describe monitor specifications in an explicit but nevertheless compact form.

%``A Dataflow Monitoring Service Based on Runtime Verification for AUTOSAR OS: Implementation and Performances''~
%\begin{itemize}
%  \item based on manual specified FSM, translated in LTL specification, generates Moore machines and then C code( mapping of system model and Moore machines), focused on temporal aspects
%  \item domain specific (no instrumentation (of SW-Cs, but the RTOS kernel)) plugged into RTOS kernel
%  \item each send and receive operation is intercepted and forwarded to a monitoring service
%\end{itemize}
%We:
%\begin{itemize}
%  \item higher abstraction level for modelling
%  \item Monitor at software component level
%  \item only necessary events are intercepted by instrumentation
%\end{itemize}

