Fig.~\ref{fig:schema} shows the architecture of our solution. The developer or designer graphically defines
the services using {\em Umbra Designer}. We have built this tool as an Eclipse plugin, using Graphiti~\cite{Graphiti} 
for the visual part, EMF~\cite{EMF} as modelling technology, and the Epsilon Generation Language (EGL)~\cite{EGL} 
for code generation. After validating the service, the tool generates a Maven project with code synthesized from
the service. The code makes use of the {\em Umbra framework}, as explained in the previous section. Once compiled, 
the project can be deployed as-is on SLEE servers, like the {\em open cloud}'s Rhino application server~\cite{rhino}.

\figeps{schema}{0.48}{Schema of {\em Umbra Designer}.}

Next, we introduce the main elements of the tool. Fig.~\ref{fig:servicio1} shows a screenshot with an example service. 
The tool abstracts services in the form of state machines, accommodating the State design pattern, as explained above.

The main canvas contains the description of a simple service. The service initial state is {\em Init}, where the 
service waits for incoming calls once a new session has been established. Hence, at the top level, the event from the 
initial state is {\em SessionRequest} (see arrow coming into state {\em Init}). The service designer does not 
have to take care of handling the {\em Bootstrap} event, as the tool itself will generate code to register the 
listeners for the events and actions used by the service (in the example all are {\em IVR} events), and identifying 
the needed resources. Events are depicted in blue (bold) over the arrows, while actions are shown in red below 
the events. The service plays a welcome message to each incoming call. This is modelled by a {\em Connected} 
event and the associated {\em PlayCollect} action. This action has the additional effect to demand pressing 
some key on the phone keypad. Thus, the service waits in state {\em KeyRequest} until the reception of some key 
stroke (event {\em Collected}). If the key pressed was ``1'' or ``2'', the service plays a different message 
in each case, as indicated by the {\em Play} action. Once the message associated to ``1'' or ``2'' is played 
(event {\em Played} in the transition going out from {\em Bye}), the service ends the call through the {\em Hangup} 
action.

\figeps{servicio1}{0.3}{Defining a simple service with {\em Umbra Designer}. }

The palette to the right contains the different types of states, transitions (i.e. events) and actions that can
appear in services. The Properties view at the bottom allows configuring any item selected in the model; in the
figure, it shows the configuration properties of the service and its resources. The tool has a contextual menu 
to validate the service (the detected warnings and errors are listed in the Problems view at the bottom), to 
generate Java code from the service, and to manage the generated Maven project (the project is shown in the 
Project Explorer tab to the left). 

Altogether, the tool promotes an agile way to work, providing support for short cycles of modelling -- 
code (re-)generation -- deployment in an integrated environment. For instance, once code is generated
for a service, developers can provide additional Java classes for further functionality 
(e.g. database persistence, which is not currently supported by our tool). We believe that this approach to 
agile modelling will be better accepted in development processes as it provides immediate value by code 
generation and service validation, and it is seamlessly integrated in the developer Java environment.

The abstract syntax of the services built with our tool is defined through a meta-model, of which
Fig.~\ref{fig:UmbraDesignerMetaModel} shows an excerpt. A state machine is configured through a 
number of properties (class {\em Properties}), like the addresses of the application and server, and 
the protocols used, among others. It is also possible to declare variables that will be available in
Java actions. Two different types of variables are supported: shared variables (object attributes), 
used to pass information between different service states, and static variables (class attributes), 
which retain their value between different service invocations. 

\figeps{UmbraDesignerMetaModel}{0.5}{Excerpt of the {\em Umbra Designer} meta-model.}

We consider five types of states: initial, final, simple, composite and choice. Each state machine has 
one initial and one final states. Composite states enable hierarchical structuring of the machine, and 
contain a reference to a state machine which can be defined within the same model or externally in a 
separate file. Choice states have multiple output branches with a boolean condition each.

Transitions can be of different types that correspond to event types of the {\em Umbra framework}. 
They are organized in six categories: Interactive Voice Response (IVR) events, Call Control events, 
HTTP (reception of HTTP requests), SMS, Text-to-Speech (TTS) and Speech recognitions events 
(for clarity, Fig.~\ref{fig:UmbraDesignerMetaModel} only shows IVR events). There are also general 
facilities, like timers. The initial event in a state machine is always of type {\em SessionRequest}.
IVR events are concerned with playing and recording media streams or with key pressing in the phone keypad. 
Examples of supported IVR events include {\em Played} (fired when a media stream finishes playing), 
{\em Recorded} (when a recording finishes) and {\em Collected} (when the user presses some key). 
Call control events include those related to the connection, disconnection and transfer of call legs. SMS 
events are concerned with the reception of text messages. Finally, TTS events are those generated by 
a TTS engine, like the start, finish, pause and resume of a speech.

A transition may have associated a sequence of actions, to be executed when the transition is triggered. 
Actions rely on the {\em Umbra} API, and are organized in similar categories to those for events. A 
special action {\em JavaCode} permits adding Java code for more general actions not directly supported 
by the framework, where it is possible to use the declared shared and static variables. The parameters 
of each action can be configured through a properties panel, as shown to the bottom of Fig.~\ref{fig:repairingService} 
for the case of the {\em PlayRecord} action.

{\em Umbra Designer} supports hierarchical modelling by means of composite states (see {\em MembersMenu}
in Fig.~\ref{fig:hierarchy}), and through references to state machines defined in other diagrams (see state 
{\em Registration} in the same figure). In the last case, a contextual menu allows opening the referenced 
machine in a different tab, as well as expanding or occluding the diagram inside the state. This feature enables
the construction of repositories of services, which then can be used to build other more complex services.

\figeps{hierarchy}{0.3}{Hierarchical modelling, and report of errors/warnings, in {\em Umbra Designer}.}

The tool enables simple validations of services prior to code generation, and displays the detected errors 
and warnings in the Eclipse Problems view (see bottom panel in Fig.~\ref{fig:hierarchy}, which contains
the detected list of errors for an information and registration service for a gym). For example, the tool reports
as an error any unreachable state, any state (different from the final state) without outgoing transitions, as well 
as non-existing paths between the initial and final states. Warnings concern the order in which some events and 
actions should occur, as some actions trigger the future occurrence of events. For example, the tool gives a 
warning if a {\em Played} event is declared, but there is no previous {\em Play} action. Notice that all these 
errors and warnings would be difficult to detect statically, if a direct encoding of the service in Java is used. 
However, we do not currently analyse {\em JavaCode} actions in transitions.

Once the model of the service is validated, it is possible to synthesize Java code from it. The tool creates a Maven 
Eclipse project, which can be deployed in an SLEE server. The generated Java code follows the State pattern~\cite{Gamma}, 
and reflects the hierarchical constructs introduced in the model. Moreover, the code includes protected regions, 
so that if the developer modifies the code manually, this is not overwritten when code is regenerated again from 
the model. 
