We are witnessing an exponential grow in the capabilities of mobile phones and in the functionality demanded by 
their users. The usual approach to deliver mobile services nowadays is the development of {\em apps} 
for a particular technology (iOS, Android, BlackBerry), running in the device of the client. This business model is 
supported by small software firms that need to develop innovative solutions in short times, in order to cope with 
an increasingly competing environment. This model has the drawback of the fragmentation of the mobile platforms, 
which implies different developments for each platform. Moreover, the functionality offered by different phones varies, 
from traditional landline phones to smartphones of the latest technology. Thus, companies may lose clients if they 
target particular platforms or assume some functionality for running their {\em app}~\cite{5640901}. 

Instead, a different alternative is to build services that do not run on the phone, but on the provider infrastructure
through dedicated servers or the cloud, and which are accessed through phone calls~\cite{Almira}. This model has the 
advantage that is independent of the mobile phone platform used, and to a certain extent, of the phone capacities. While 
these ``server-side'' services cannot replace all types of phone {\em apps}, they are useful for many scenarios. They
are normally driven by voice and DTMF key strokes. Examples include voice notes and voice-to-email services; 
services to inject customized background sounds in phone calls; the customization of the telephone keys to inject 
``voice smileys'' in a conversation; as well as the typical Interactive Voice Response (IVR) applications of call 
centers, telebanking, credit card services, and so on. These services tend to require a reduced customer learning curve
compared to those offered by mobile {\em apps}.

JAIN SLEE~\cite{JAINSLEE} is a standard of the Java Community Process for developing event-based 
telecommunication applications, which can be used to build server-side telephony services. Applications 
developed with JAIN SLEE can be deployed on any server implementing it. A Service Logic Execution 
Environment (SLEE) is an efficient event processing application environment with high throughput and 
low latency. A service is built by the construction and interconnection of components, and their subsequent
deployment in SLEE servers.

Almira Labs has developed the {\em Umbra framework}, a Java API that leverages 
on the JAIN SLEE standard. The framework simplifies the development of JAIN SLEE applications by providing 
a higher-level view of the event flows and protocols involved in a telecommunications application, and provides 
true portability across different SLEE implementations. 
Still, using this framework requires specialized knowledge in JAIN SLEE and Java. In order to make service 
construction possible for non-experts -- namely, people from customer companies -- Almira and some 
researchers of the Universidad Aut\'onoma have developed {\em Umbra Designer}, a tool for the graphical
development of telephony services. The tool abstracts services in the form of hierarchical state machines using the 
events and actions available in the {\em Umbra framework}. 
The tool integrates a code generator that produces a {\em Maven}~\cite{Maven} Java project, which 
can be deployed as-is in JAIN SLEE servers for execution. The aim of the tool is to facilitate 
service modelling to non-experts in the API, and speed-up the development for programmers. This is 
demonstrated by a series of experiments where we have measured the efficiency of manual service 
development using the Java API with respect to using the tool, reporting an increase of productivity of 
more than 40\% in the average case.

\paragraph{Paper organization.} Section~\ref{sec:api} provides some background on 
JAIN SLEE and the {\em Umbra framework}. Section~\ref{sec:designer} introduces the graphical 
modelling tool. Section~\ref{sec:evaluation} presents the evaluation results. Section~\ref{sec:related} 
compares with related work, and finally, Section~\ref{sec:conclusions} ends with the conclusions and plans 
for future work.
