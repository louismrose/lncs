Model typing brings the benefit associated with well-defined type systems to model-driven development (MDD) through the assignment of specific types to models. 
In particular, model type systems enable reuse of model manipulation operations (e.g., model transformations), where manipulations defined for models typed by a supertype can be used to manipulate models typed by subtypes. 
Existing model typing approaches are limited to structural typing defined in terms of object-oriented metamodels (e.g., MOF), in which the only structural (well-formedness) constraints are those that can be expressed directly in metamodeling notations (e.g., multiplicity and element containment constraints).
In this paper we describe an extension to model typing that takes into consideration structural invariants, other than those that can be expressed directly in a metamodeling notation, and specifications of behaviors associated with model types. 
The approach supports contract-aware substitutability, where contracts are defined in terms of invariants and pre-/post-conditions expressed using OCL. 
Support for behavioral typing paves the way for behavioral substitutability.
We also describe a technique to rigorously reason about model type substitutability as supported by contracts, and apply the technique in a usage scenario from the optimizing compiler community.


\begin{comment}
Model typing extends the applicability of typing to model-oriented type
system by assigning models with specific types. It provides support for
model substitutability addressing a wide range of facilities such as model
transformation reuse. While existing approaches are limited to
object-oriented metamodels (e.g., MOF) as types, there is a need for
exploring more precise types. In particular, we propose in this paper an
extension to model typing that takes into account contract-aware
substitutability where contracts are defined in terms of invariants and
pre-/postconditions expressed using OCL. While invariants offer a suitable
way to complete object-oriented metamodels with additional structural
properties, pre-/postconditions pave the way of behavioral
substitutability for model transformation specialization. We also provide
an implementation to rigorously reason about the substitutability on model
types with contracts and apply it on use cases coming from the optimizing
compiler community.
\end{comment}