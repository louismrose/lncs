The need for developing and making available telecommunication APIs, is discussed in~\cite{5640901}. Similar to our rationale, the author
foresees the possibility of telecommunication application stores -- similar to those of Apple and Android -- based on the availability of service creation environments. 
%While most large companies offer different forms of service creation environments and deployment frameworks, they are closed systems. (?)

There are several implementations of the JAIN SLEE, like Mobicents~\cite{Mobicents} and
OpenCloud~\cite{rhino}. The latter includes a visual builder for services, the
Visual Service Architect (VSA)~\cite{VSA}. In this environment, a service is described by an application-scenario diagram (to configure properties, protocols and
resources), state machine diagrams (to describe service states) and flowchart diagrams (to describe actions). VSA targets general JAIN SLEE 
services, not necessary for telephony, and hence lacks high-level constructs (both events and actions) for voice-driven telephony services, as we 
provide in {\em Umbra Designer}. VSA state
machines and flowcharts tend to be of lower level of abstraction due to the lack of constructs like hierarchical states, choice states and key strokes
branches, among others.

In~\cite{ICT}, the authors present an environment for service composition using MetaEdit+. SBBs are programmed 
in Java, which become reusable and can be composed graphically. Our approach is different as the blocks themselves 
are modelled using state machines, from which Java code is generated. Also in the context of MetaEdit+, in~\cite{DSVLs}, 
the authors describe a graphical language to define simple call processing services. This language allows defining the 
flow for handling incoming calls like rerouting them, or sending a message upon their reception. The services can be 
serialized in XML. In our case, state machines are a better abstraction for the event-driven nature of voice-driven 
telephony services, while we need to generate more complex Java code. Another language for telephony service 
creation is SPL~\cite{palix:inria-00196520}, a scripting textual language with formal semantics. It differs from our
approach in that it is targeted to experienced programmers, and its formal semantics enables critical properties of services 
to be guaranteed. We plan to address exhaustive testing of service models against user actions in future work.

VoiceXML~\cite{VoiceXML} is a W3C standard to describe interactive voice dialogues between a human and a computer. VoiceXML files
are played by voice browsers, and contain tags that instruct the browser to provide speech synthesis, automatic speech recognition, 
dialog management, and audio playback. VoiceXML applications are accessed via HTTP, while we use phone protocols.

On a final comment, there are not many published results of efficiency of MDE in practice~\cite{HutchinsonWRK11,DSVLs}. 
Our work also contributes in this direction, by describing a specific successful scenario for the applicability of MDE.

