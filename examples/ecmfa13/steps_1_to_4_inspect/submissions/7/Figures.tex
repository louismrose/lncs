%Figure makros for one column and two column layouts 

\usepackage{graphicx}
%\usepackage{sideways}
\usepackage{rotate}
% Trennungsstrich aus der Diss


% Alter some LaTeX defaults for better treatment of figures:
    %http://mintaka.sdsu.edu/GF/bibliog/latex/floats.html
    % See p.105 of "TeX Unbound" for suggested values.
    % See pp. 199-200 of Lamport's "LaTeX" book for details.
    %   General parameters, for ALL pages:
    \renewcommand{\topfraction}{0.9}	% max fraction of floats at top
    \renewcommand{\bottomfraction}{0.8}	% max fraction of floats at bottom
    %   Parameters for TEXT pages (not float pages):
    \setcounter{topnumber}{2}
    \setcounter{bottomnumber}{2}
    \setcounter{totalnumber}{4}     % 2 may work better
    \setcounter{dbltopnumber}{2}    % for 2-column pages
    \renewcommand{\dbltopfraction}{0.9}	% fit big float above 2-col. text
    \renewcommand{\textfraction}{0.07}	% allow minimal text w. figs
    %   Parameters for FLOAT pages (not text pages):
    \renewcommand{\floatpagefraction}{0.7}	% require fuller float pages
	% N.B.: floatpagefraction MUST be less than topfraction !!
    \renewcommand{\dblfloatpagefraction}{0.7}	% require fuller float pages


\newcommand\Sidebar[2]{
\noindent\framebox[\textwidth][l]{
\begin{minipage}[t]{0.96\textwidth}
\subsection*{#1}
#2\end{minipage}
}}

\newcommand\FloatingSidebar[2]{
\begin{figure*}[tbp]
\framebox[\textwidth][l]{
\begin{minipage}[t]{0.96\textwidth}
\subsection*{#1}
#2\end{minipage}
\vspace{-48pt}
}\end{figure*}
}

\newcommand\FloatingSidebarI[2]{
\begin{figure}[tbp]
\framebox[\columnwidth][l]{
\begin{minipage}[t]{0.96\columnwidth}
\subsection*{#1}\vspace{2mm}
#2\end{minipage}
%\vspace{-48pt}
}\end{figure}
}

\newcommand{\Separator}{\\\vspace*{2mm}\rule{\textwidth}{0.3pt}\\\vspace*{5mm}}

%\newcommand\FFF[1]{{Figure~\ref{fig:#1}}}
\newcommand\FFF[1]{{Fig.~\ref{fig:#1}}}
\newcommand\SSS[1]{{Section~\ref{sec:#1}}}
\newcommand\CCC[1]{{Chapter~\ref{sec:#1}}}
\newcommand\TTT[1]{{Table~\ref{tab:#1}}}
\newcommand\SEE[1]{{(see \FFF{#1})}} % obsolete - for backward compatibility only
\newcommand\SEEF[1]{{(see \FFF{#1})}}
\newcommand\SEES[1]{{(see \SSS{#1})}}
\newcommand\SEET[1]{{(see \TTT{#1})}}



% Einspaltige Abbildung

\newcommand{\SKETCH}[1]{\begin{center}\includegraphics[width=\columnwidth]{#1}\end{center}}

\newcommand{\SFG}[3]{\Figure{\includegraphics[width=#1\columnwidth]{#2}}{#3}{fig:#2}}
\newcommand{\FG}[2]{\FIG{#1}{#2}{fig:#1}}
\newcommand{\FIG}[3]{\Figure{\includegraphics[width=\columnwidth]{#1}}{#2}{#3}}
\newcommand{\Figure}[3]{
\begin{figure}[htbp]%abovecaptionskip=0pt
%\rule{\textwidth}{1pt}
\begin{center}#1\end{center}
%rule{\textwidth}{0.3pt}
%\vspace{-8pt}% nur f�r MQ-3!! und MQ-4
\caption{#2}\label{#3}
%\vspace{-6pt}% nur f�r MQ-3!!
%vspace{-5pt}\rule{\textwidth}{1pt}
%\vspace{-6pt}% nur f�r MQ-3!! und MQ-4 und MQ-4b
\end{figure}
}



% Zweispaltige Abbildung (z.B. IEEE-Layout)

\newcommand{\BSKETCH}[1]{\begin{center}\includegraphics[width=\textwidth]{#1}\end{center}}

\newcommand{\BFG}[2]{\BFIG{#1}{#2}{fig:#1}}
\newcommand{\BFIG}[3]{\BroadFigure{\includegraphics[width=\textwidth]{#1}}{#2}{#3}}
\newcommand{\BroadFigure}[3]{
\begin{figure*}[!htbp]%abovecaptionskip=0pt
%\rule{\textwidth}{1pt}
\begin{center}#1\end{center}%vspace{-12pt}
%rule{\textwidth}{0.3pt}
%\vspace{-14pt}% nur f�r MQ-3!!
\caption{#2}\label{#3}
%\vspace{-6pt}% nur f�r MQ-3!!
%\vspace{-5pt}\rule{\textwidth}{1pt}
\end{figure*}}


% Quer-Abbildung
\newlength{\FOOheight}
\setlength{\FOOheight}{\textheight}
\addtolength{\FOOheight}{-3.5cm}


% braucht nur rotating, was ich standardm��ig dabeihabe
%\begin{sidewaystable}
%  \centering \EVALUATIONTABLE
%  \caption{Approaches to querying models, ordered by decreasing overall score.}\label{tab:comparison}
%\end{sidewaystable}

\newcommand{\WideFigure}[3]{
\begin{figure}[tbp]
%\rule{\textwidth}{1pt}
\begin{center}
\includegraphics[width=\FOOheight,angle=90]{#1}
\end{center}
\rule{\textwidth}{0.3pt}
\caption{#2}\label{#3}
%\vspace{-5pt}\rule{\textwidth}{1pt}
\end{figure}}


% Abbildung mit Latex-Text

\newcommand{\TextFigure}[3]{
\begin{figure}[!htbp]\abovecaptionskip=0pt
%\rule{\textwidth}{1pt}
\begin{center}#1\end{center}
\vspace{-12pt}\rule{\textwidth}{0.3pt}
\caption{#2}\label{#3}
\vspace{-24pt}\rule{\textwidth}{1pt}
\end{figure}}


% - - - - - - - - -  T A B L E S  - - - - - - - - - - - - - - - 

%IEEE-Tabelle (Beschriftung oben)

\newcommand{\Table}[3]{
\begin{table}[tbp]
%\renewcommand{\arraystretch}{1.3}
%\vspace{-12pt}% nur f�r MQ-3!! und MQ-4
\caption{#2}\label{tab:#3}
%\vspace{-6pt}% nur f�r VL-4c
%\vspace{-8pt}
\begin{center}#1\end{center}
%\vspace{-6pt}% nur f�r MQ-3!!
%\vspace{-12pt}
\end{table}}

%IEEE-Tabelle (Beschriftung oben, zweispaltig)

\newcommand{\TableI}[3]{
\begin{table*}[tbp]
%\renewcommand{\arraystretch}{1.3}
%\vspace{-12pt}% nur f�r MQ-3!! und MQ-4
\caption{#2}\label{tab:#3}
\vspace{-8pt}
\begin{center}#1\end{center}
\vspace{-6pt}% nur f�r MQ-3!!
%\vspace{-12pt}
\end{table*}}


% Quer-Tabelle (einspaltig)

\newcommand{\WideTableI}[3]{
\begin{table}[!htbp]
%\rule{\textwidth}{1pt}
\rotatebox{90}{\begin{center}
#1
\end{center}}
%\rule{\textwidth}{0.3pt}
\caption{#2}\label{#3}
%\vspace{-5pt}\rule{\textwidth}{1pt}
\end{table}}


% Quer-Tabelle (zweispaltig)
\newcommand{\WideTable}[3]{
\begin{table*}[!htbp]
%\rule{\textwidth}{1pt}
\rotatebox{90}{\begin{center}
#1
\end{center}}
\rule{\textwidth}{0.3pt}
\caption{#2}\label{#3}
\vspace{-5pt}\rule{\textwidth}{1pt}
\end{table*}}


% Breite L�ngstabelle (zweispaltig)

\newcommand{\MediumWideTable}[3]{
\begin{table*}[!htbp]
%\rule{\textwidth}{1pt}
\begin{center}{\small #1}\end{center}
%\rule{\textwidth}{0.3pt}
\caption{#2}\label{#3}
%\vspace{-5pt}\rule{\textwidth}{1pt}
\end{table*}}
